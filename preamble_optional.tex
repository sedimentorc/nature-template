%%%%%%%%%%%%%%%%%%%%%%%%%%%%%%%%%%%%%%
%% OPTIONAL
%%%%%%%%%%%%%%%%%%%%%%%%%%%%%%%%%%%%%%

%extra packages: todonotes, ulem, rotating, soul, color
%special commands: \NEW, \NOTE, \NOTEC, \NOTEM, \Conv, \xoverline

% optional special text packages
\usepackage{ulem} %provides various types of underlining that can stretch between words and be broken across lines
\usepackage{rotating} % rotate any object by arbitrary angle
\usepackage{soul}%h y p h e n -a t a b l e l e t t e r s p a c i n g ( s p a c i n g o u t ) , underlining and some derivatives such as overstriking and highlighting


% tools to edit with multiple authors
\usepackage{color} %change color of text

% use new to ask people to highlight changes
\newcommand{\NEW}[1]{{\color{red}{#1}}}

%use todonotes to specify notes and todonotes
\usepackage[colorinlistoftodos]{todonotes}


\newcommand{\NOTEM}[2][]
{\todo[caption={#2}, size=\small, #1,color=red!40]{\renewcommand{\baselinestretch}{0.5}\selectfont#2\par}}

\newcommand{\NOTEC}[2][]
{\todo[caption={#2}, size=\small, #1,color=blue!40]{\renewcommand{\baselinestretch}{0.5}\selectfont#2\par}}

\newcommand{\NOTE}[2][]
{\todo[caption={#2}, size=\small, #1]{\renewcommand{\baselinestretch}{0.5}\selectfont#2\par}}

% consider automatically making NOTE be an inline function if inside float. 
%you can detect if inside float using a simple if statement
%http://tex.stackexchange.com/questions/27172/how-can-i-detect-if-im-inside-or-outside-of-a-float-environment




%define convolution symbol
\newcommand{\Conv}{\mathop{\scalebox{1.5}{\raisebox{-0.2ex}{$\ast$}}}}%

%http://tex.stackexchange.com/questions/22100/the-bar-and-overline-commands
%provides wide overline command
% \xoverline[width percent]{symb}
% note, it doesn't scale correctly inside superscripts
\makeatletter
\newsavebox\myboxA
\newsavebox\myboxB
\newlength\mylenA

\newcommand*\xoverline[2][0.75]{%
    \sbox{\myboxA}{$\m@th#2$}%
    \setbox\myboxB\null% Phantom box
    \ht\myboxB=\ht\myboxA%
    \dp\myboxB=\dp\myboxA%
    \wd\myboxB=#1\wd\myboxA% Scale phantom
    \sbox\myboxB{$\m@th\overline{\copy\myboxB}$}%  Overlined phantom
    \setlength\mylenA{\the\wd\myboxA}%   calc width diff
    \addtolength\mylenA{-\the\wd\myboxB}%
    \ifdim\wd\myboxB<\wd\myboxA%
       \rlap{\hskip 0.5\mylenA\usebox\myboxB}{\usebox\myboxA}%
    \else
        \hskip -0.5\mylenA\rlap{\usebox\myboxA}{\hskip 0.5\mylenA\usebox\myboxB}%
    \fi}
\makeatother

%% define dummy figure
% make a figure by calling \dummyfig{Dummy Figure Label} instead of includegraphics in the figure environment
% see https://gist.github.com/dpgettings/9635856
\newcommand{\dummysinglecolfig}[1]{
  \centering
  \fbox{
    \begin{minipage}[c][89mm][c]{89mm}
      \centering{#1}
    \end{minipage}
  }
}
 
\newcommand{\dummydoublecolfig}[1]{
  \centering
  \fbox{
    \begin{minipage}[c][89mm][c]{183mm}
      \centering{#1}
    \end{minipage}
  }
}
  



